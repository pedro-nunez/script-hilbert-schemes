\documentclass[12pt,a4paper]{amsart}

\usepackage[T1]{fontenc}
\usepackage[utf8]{inputenc}
\usepackage[british]{babel}
\usepackage{mathtools}
\usepackage{amsthm}
\usepackage{amssymb}
\usepackage{mathrsfs}
\usepackage{enumitem}
\usepackage{tikz-cd}
\usetikzlibrary{decorations.markings}
\usepackage{float}
\usepackage{hyperref}
\urlstyle{same}
\usepackage[noabbrev]{cleveref}

\theoremstyle{plain}
\newtheorem{thm}{Theorem}
\newtheorem*{thm*}{Theorem}
\newtheorem{lm}[thm]{Lemma}
\newtheorem{prop}[thm]{Proposition}
\newtheorem{cor}[thm]{Corollary}
\theoremstyle{definition}
\newtheorem{defn}[thm]{Definition}
\newtheorem{exmp}[thm]{Example}
\newtheorem{xca}[thm]{Exercise}
\theoremstyle{remark}
\newtheorem{rem}[thm]{Remark}
\Crefname{thm}{Theorem}{Theorems}
\Crefname{lm}{Lemma}{Lemmas}
\Crefname{prop}{Proposition}{Propositions}
\Crefname{cor}{Corollary}{Corollaries}
\Crefname{defn}{Definition}{Definitions}
\Crefname{exmp}{Example}{Examples}
\Crefname{xca}{Exercise}{Exercises}
\Crefname{rem}{Remark}{Remarks}

\title[Talk on Hilbert schemes of points on surfaces]{Talk on Hilbert schemes of points on surfaces}
\author[Pedro N\'{u}\~{n}ez]{Pedro N\'{u}\~{n}ez}
\address{Pedro N\'{u}\~{n}ez \newline
\indent Albert-Ludwigs-Universit\"{a}t Freiburg, Mathematisches Institut \newline
\indent Ernst-Zermelo-Straße 1, 79104 Freiburg im Breisgau (Germany)}
\email{\normalfont\href{mailto:pedro.nunez@math.uni-freiburg.de}{pedro.nunez@math.uni-freiburg.de}}
\renewcommand*{\urladdrname}{\itshape Homepage}
\urladdr{\normalfont\href{https://home.mathematik.uni-freiburg.de/nunez/}{https://home.mathematik.uni-freiburg.de/nunez}}
\thanks{The author gratefully acknowledges support by the DFG-Graduiertenkolleg GK1821 ``Cohomological Methods in Geometry'' at the University of Freiburg.}
\date{\today}

\setcounter{tocdepth}{1}
\sloppy
\makeatletter
\hypersetup{
  pdfauthor={\authors},
  pdftitle={\@title},
  colorlinks,
  linkcolor=[rgb]{0.2,0.2,0.6},
  citecolor=[rgb]{0.2,0.6,0.2},
  urlcolor=[rgb]{0.6,0.2,0.2}}
\makeatother

\begin{document}

\maketitle

\begin{abstract}
  Script for the 7\textsuperscript{th} talk of the seminar on Heisenberg algebras and Hilbert schemes of points on surfaces organized by Mara Ungureanu during the Summer Term 2021 at the University of Freiburg.
\end{abstract}

\tableofcontents

\begin{center}
  \textcolor{gray}{---parts in gray will be omitted during the talk---}
\end{center}

\setcounter{section}{-1}

\section{Conventions and notation}

We always work over $\mathbb{C}$.
By a variety we mean an integral separated scheme of finite type over $\mathbb{C}$.

\appendix

\section{Quotients of quasi-projective varieties by finite groups}

We will mostly follow the notes in \url{http://www.math.lsa.umich.edu/~mmustata/appendix.pdf} in this appendix.

\begin{lm}\label{lm:finitetype}
  Let $G$ be a finite group.
  Let $A$ be a finite type $\mathbb{C}$-algebra and assume that the group $G$ acts on $A$ from the left by $\mathbb{C}$-algebra automorphisms.
  Then the set of invariant elements $A^{G}$ is a $\mathbb{C}$-subalgebra of $A$ which is of finite type over $\mathbb{C}$.

  \begin{proof}
    Let $\rho \colon G \to \operatorname{Aut}_{\mathbb{C}}(A)$ be the given left action.
    Let us first quickly ensure that
    \[ A^{G} := \bigcap_{g \in G} \{ a \in A \mid \rho(g)(a) = a \} \]
    is a $\mathbb{C}$-subalgebra of $A$.
    \begin{itemize}
      \item $A^{G} \subseteq A$ is a subgroup.
        Indeed, since $\rho(g)$ is a ring morphism for every $g \in G$, we have $0 \in A^{G}$.
        And if $a_{1}, a_{2} \in A^{G}$ and $g \in G$, then it follows again from $\rho(g)$ being a ring morphism that
        \[ \rho(g)(a_{1}+a_{2}) = \rho(g)(a_{1}) + \rho(g)(a_{2}) = a_{1} + a_{2}. \]
      \item $A^{G} \subseteq A$ is a subring.
        We have seen already that it is a subgroup.
        Since $\rho(g)$ is a ring morphism for every $g \in G$, we also have $1 \in A^{G}$, so it remains only to show that $A^{G}$ is closed under products.
        If $a_{1}, a_{2} \in A^{G}$ and $g \in G$, then using once again that $\rho(g)$ is a ring morphism we see that
        \[ \rho(g)(a_{1}a_{2}) = \rho(g)(a_{1})\rho(g)(a_{2}) = a_{1}a_{2}. \]
      \item $A^{G} \subseteq A$ is a $\mathbb{C}$-vector subspace.
        We have seen already that it is a subgroup, so it remains only to show that $A^{G}$ is closed under scalar product.
        If $a \in A^{G}$, $\lambda \in \mathbb{C}$ and $g \in G$, then we use the assumption that $\rho(g)$ is $\mathbb{C}$-linear to deduce that
        \[ \rho(g)(\lambda a) = \lambda \rho(g)(a) = \lambda a. \]
    \end{itemize}
    
    The other assertion in the lemma is that $A^{G}$ is a finite type $\mathbb{C}$-algebra.
    The idea is to write $A^{G}$ as a finite $B$-module for some suitable finite type $\mathbb{C}$-algebra $B$.
    Then it would follow that $A^{G}$ is a finite type $\mathbb{C}$-algebra as well.
    Indeed, let $\beta_{1}, \ldots, \beta_{m} \in B$ be generators of $B$ as an algebra over $\mathbb{C}$, and let $e_{1}, \ldots, e_{l} \in A^{G}$ be generators of $A^{G}$ as a $B$-module.
    Then we can write any $a \in A^{G}$ as a $B$-linear combination
    \[ a = \sum_{i = 1}^{l} b_{i}e_{i}, \]
    and in turn each $b_{i}$ as an algebraic combination
    \[ b_{i} = f_{i}(\beta_{1}, \ldots, \beta_{m}) \]
    for some $f_{i} \in \mathbb{C}[\beta_{1}, \ldots, \beta_{m}]$.
    It follows that we can write $a$ as an algebraic combination in the variables $\beta_{1}, \ldots, \beta_{m}, e_{1}, \ldots, e_{l}$, so these elements would form a system of generators of $A^{G}$ as a $\mathbb{C}$-algebra.

    In order to construct such $B$, we first note that the inclusion $A^{G} \subseteq A$ is an integral ring extension.
    Indeed, every $a \in A$ is a root of the monic polynomial
    \[ P_{a}(t) := \prod_{g \in G}(t - \rho(g)(a)), \]
    whose coefficients are in $A^{G}$ by Vieta's formulas.
    Let $\alpha_{1}, \ldots, \alpha_{m} \in A$ be generators of $A$ as an algebra over $\mathbb{C}$.
    Let $\{ c_{i,j} \}_{j = 0}^{d_{i}}$ be the coefficients of $P_{\alpha_{i}}$ for each $i \in \{1, \ldots, m\}$.
    Then define $B$ to be the $\mathbb{C}$-subalgebra of $A$ generated by all these coefficients $\{ c_{1,0}, \ldots, c_{1,d_{1}}, c_{2,0}, \ldots, c_{m,d_{m}} \}$.
  \end{proof}

\end{lm}

\bibliographystyle{alpha}
\bibliography{main}
\vfill

\end{document}
